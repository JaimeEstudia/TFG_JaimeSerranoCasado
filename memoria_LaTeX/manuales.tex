\chapter{Manual de Instalaci\'{o}n}

Este cap\'{\i}tulo se dedica a explicar los prerrequisitos t\'{e}cnicos, sistema operativo, etc, necesarios para instalar la aplicaci\'{o}n desarrollada en el TFG. 

Se explicar\'{a} tambi\'{e}n todo lo necesario para conseguir e instalar el software de terceros que se precise con car\'{a}cter previo. 

Si se trata de una aplicaci\'{o}n WEB, explicar c\'{o}mo instalar y configurar los servidores necesarios para que funcione dicha aplicaci\'{o}n, 
y c\'{o}mo se realiza el despliegue de la misma. 

Si se trata de una aplicaci\'{o}n de escritorio, explicar el proceso de instalaci\'{o}n y configuraci\'{o}n en las plataformas correspondientes. 

\chapter{Manual de Usuario}

Este cap\'{\i}tulo final se dedica a desarrollar el manual del usuario final, as\'{\i} como del usuario administrador, si fuese necesario. 

\section{Manual de Usuario}

Se recomienda que contenga la informaci\'{o}n relevante, lo m\'{a}s completa y clara posible, como si fuese el manual de un producto comercial, 
dirigido a un usuario no avanzado. Esta documentaci\'{o}n pudiera coincidir (de hecho, se recomienda) con la AYUDA del programa. 

\section{Manual de Administraci\'{o}n}

Este manual ser\'{\i}a mucho m\'{a}s t\'{e}cnico, pero asimismo igual de claro y completo, y estar\'{\i}a dirigido al administrador de la aplicaci\'{o}n. 

