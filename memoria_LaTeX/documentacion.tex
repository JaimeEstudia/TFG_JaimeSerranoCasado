\chapter{An\'{a}lisis}

\section{Requisitos}

En este apartado se describen los requisitos del sistema que se va a desarrollar: requisitos funcionales, de interfaz de usuario, de informaci\'{o}n, etc. 
Asimismo, se especificar\'{a} si la aplicaci\'{o}n debe ser o no multiplataforma (o multiling\"{u}e), si se establecen restricciones de uso o de otro tipo 
(por ejemplo heredadas del entorno organizativo en el que se integrar\'{a}), etc. 

Para una mejor descripci\'{o}n de los requisitos, se usan los llamados casos de uso, basados en la identificaci\'{o}n de actores y tareas. 

A modo de ejemplo, este ser\'{\i}a un caso de uso para la tarea \lq\lq solicitar registro\rq\rq \ en una aplicaci\'{o}n WEB, que deba ser validado por un administrador: 

\begin{center}
\begin{arial}
\renewcommand{\arraystretch}{1.3}
\begin{longtable}{|l|p{10cm}|}
\hline
\textbf{Nombre e ID del CU} & CU-01. Solicitar registro \\ \hline
\textbf{Actor} & Usuario \\ \hline
\textbf{Descripci\'{o}n} & El usuario env\'{\i}a sus datos de usuario al sistema, generando una solicitud de alta que deber\'{a} ser gestionada por un administrador. \\ \hline
\textbf{Precondiciones} & PRE-1. El usuario no est\'{a} identificado en el sistema. \\ \hline
\textbf{Postcondiciones} & POST-1. La solicitud queda almacenada en el sistema. \\ \hline
\textbf{Flujo normal} &
\begin{enumerate}[label=FN\arabic*]
\item El actor introduce sus datos de usuario e indica al sistema que quiere solicitar el registro. 
\item El sistema comprueba los datos introducidos.
\item Si los datos son correctos, el sistema almacena la solicitud de registro e informa al actor del resultado. 
\end{enumerate} \\ \hline
\textbf{Flujo alternativo 1} &
\begin{itemize}
\item [FA3] Si los datos son incorrectos se informa al usuario del error y no se procede a almacenar la solicitud. 
\end{itemize} \\ \hline
\textbf{Flujo alternativo 2} &
\begin{itemize}
\item [FA3] Si no hay otros usuarios, el usuario pasa autom\'{a}ticamente al estado de activaci\'{o}n y tendr\'{a} el rol de administrador. 
\end{itemize} \\ \hline
\textbf{Excepciones} &
\begin{enumerate}[label=E\arabic*]
\item El usuario ha dejado campos requeridos sin rellenar.
\item El usuario o correo electr\'{o}nico ya existen.
\item El usuario est\'{a} bloqueado.
\end{enumerate} \\ \hline
\textbf{Prioridad} & Alta \\ \hline
\textbf{Otra info} & El primer usuario registrado en la aplicaci\'{o}n ser\'{a} el que adquiera el rol de administrador. \\ \hline
\caption{CU-01. Solicitar registro}
\label{tab:CU01}
\end{longtable}
\end{arial}
\end{center}

% Diagramas de estados y transiciones de los posibles actores ... 

\section{Atributos de calidad}

Se analizar\'{a}n aqu\'{\i} los posibles indicadores de calidad de la aplicaci\'{o}n desarrollada, a saber: 

\begin{description}

\item[Tiempo de corrección]: La aplicación deberá ser capaz de evaluar un proyecto académico de programación de hasta 2 000 líneas de código distribuidas en múltiples archivos en un tiempo máximo de 1 minuto, incluyendo el tiempo de comunicación con la API de la LLM, en un equipo local con 16 GB de RAM y CPU de 8 núcleos.

\item[Uso de memoria]: El consumo máximo de memoria de la aplicación cliente durante el proceso de corrección no deberá superar el 60\% de la memoria RAM disponible.

\item[Ejecución concurrente]: El sistema deberá permitir la evaluación concurrente de al menos dos proyectos de programación independientes mediante llamadas paralelas a la API, sin que el tiempo medio de respuesta por proyecto se incremente más de un 40\%.

\item[Privacidad de los datos]: El código fuente enviado a la API de la LLM será solo el contenido necesario para la evaluación y no se registrará ninguna informacion sensible del usuario. El codigo tampoco se almacenará de forma persistente en la aplicación cliente si este no lo aprueba.

\item[Aislamiento de archivos]: La aplicación deberá restringir el acceso a los archivos del sistema únicamente al directorio del proyecto a evaluar, impidiendo la lectura, ejecución o modificación de código fuera de dicho ámbito.

\item[Gestión de errores]: Ante fallos en la comunicación con la API de la LLM, tiempos de espera excedidos o respuestas inválidas, la aplicación deberá capturar la excepción, informar al usuario de forma clara y permitir reintentar la operación sin finalizar abruptamente la ejecución.

\item[Tolerancia a entradas inválidas]: El sistema deberá detectar y rechazar proyectos que contengan archivos no soportados, dependencias inexistentes o código corrupto, notificando el error al usuario en un tiempo máximo de 10 segundos antes de iniciar el proceso de evaluación.

\item[Estabilidad prolongada]: La aplicación deberá mantener un funcionamiento estable durante sesiones continuas de al menos 2 horas evaluando múltiples proyectos de programación consecutivos, sin fugas de memoria, acumulación de procesos de red ni degradación significativa del rendimiento.


\end{description}

\chapter{Dise\~{n}o}

Los aspectos de dise\~{n}o depende mucho del paradigma elegido (programaci\'{o}n orientada a objetos o no, por ejemplo).  \index{Programaci\'{o}n Orientada a Objetos}

\section{Dise\~{n}o de datos}

En primer lugar podemos ver el modelo de datos que se usar\'{a} para dise\~{n}ar la base de datos, si esta es necesaria.  \index{Bases de Datos}
As\'{\i}, habr\'{a} que mostrar el diagrama entidad-relaci\'{o}n, el modelo relacional, y el diccionario de datos, a nivel l\'{o}gico, 
y por otra parte el dise\~{n}o f\'{\i}sico de la base de datos, y los perfiles de acceso a la misma por parte de los usuarios (si procede). 

\section{Diagramas de clase y de secuencia}

Si la programaci\'{o}n es orientada a objetos, es necesario especificar los diagramas de clases que se usan. 

Los diagramas de secuencia describen la secuencia de pasos con que un usuario interacciona con el sistema, 
y est\'{a}n relacionados con los casos de uso de la etapa de an\'{a}lisis. 

Es interesante tambi\'{e}n especificar si se ha hecho uso de dise\~{n}o de patrones. 

\chapter{Implementaci\'{o}n}

En  este cap\'{\i}tulo se mostrar\'{a}n los detalles m\'{a}s relevantes en cuanto a la implementaci\'{o}n del sistema. 

\chapter{Pruebas}

Este cap\'{\i}tulo es muy importante, y muestra las pruebas que se han realizado para demostrar el funcionamiento del sistema. 

Se referir\'{a}n las estrategias de prueba utilizadas, y cu\'{a}ndo se han efectuados dichas pruebas. 

Se realizar\'{a}n pruebas de caja negra y de caja blanca, unitarias, de integraci\'{o}n, de rendimiento, de seguridad, etc. 

Se realizar\'{a} una buena bater\'{\i}a de pruebas para probar la funcionalidad de la aplicaci\'{o}n, que est\'{a} determinada por los requisitos de la misma, 
y se mostrar\'{a}n los resultados (si estos son muchos, se pueden dejar los menos relevantes para el contenido del CD o para los Anexos de la Memoria). 

